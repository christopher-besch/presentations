% to be run with: pdflatex -shell-escape stromwaage.tex
\documentclass[convert]{standalone}
\usepackage{tikz}
\usepackage[compatibility]{circuitikz}

% from: https://tex.stackexchange.com/questions/123760/draw-crosses-in-tikz
\usetikzlibrary{shapes.misc}
\tikzset{cross/.style={cross out, draw=white, minimum size=2*(#1-\pgflinewidth), inner sep=0pt, outer sep=0pt},
% default radius will be 1pt. 
cross/.default={1pt}}

\begin{document}

\begin{circuitikz}[color=white]
    % end of magnetic field
    \draw (-4,0) to[*short] ++(6,0);
    % magnetic field direction
    \draw (-3.5,-0.5) node[cross=4pt] {};
    \draw (-3.5,-0.5) circle (5pt);

    % power source
    \draw (-3, 1) coordinate(left) to[*short, -o] ++(0.5, 0) coordinate(minus) node[right]{$-$};
    \draw (minus) ++(1.5, 0) coordinate(right_pow) to[*short, -o] ++(-0.5, 0) node[left]{$+$};
    % ampere meter
    \draw (right_pow) to[*rmeterwa, t=$I$] (1, 1) coordinate(right);

    % circuit
    \draw (left) to[*short] ++(0, -4) to[*short] (1, -3) to[*short] (right);
    % lenght l
    \draw (-3, -4) to[*short=$l$] (1, -4);
    \draw (-3, -3.9) to[*short] (-3, -4.1);
    \draw (1, -3.9) to[*short] (1, -4.1);

    % force vector
    \draw [-stealth](0, -3) -- (0,-3.5) node[right]{$F_L$};
\end{circuitikz}

\end{document}
